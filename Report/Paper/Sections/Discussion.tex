All the circuits of 2017 are listed in table \ref{table:results}, with the corresponding data. The last column shows the amount of crashes, computed by the simulations. The number of crashes seems divided equally over the circuits. This is in line with the real-world amount of crashes this season so far.

Figure \ref{fig:results1} shows an almost horizontal line in the distance that is being used to brake before the turn. This indicates that the amount of metres that is shown on the track by brake-pads/distance indicators, seems right and doesn't affect the amount of crashes. The used model doesn't take the width of turns into account. Based on the 2017 circuits, it's clear that 2 or 3 cars next to each other is the maximum, instead of the 5 rows we used. Adding this constraint into future research will influence the way of braking.

According to figure \ref{fig:results2}, we see that the number of crashes increases when the distance of a turn calculated from start rises. There are numerous causes for this, high speed could be one. Putting it in perspective, you could also say that there's more metres to crash, so it's logical that the number gets higher. However, the distance of the turn doesn't say anything about the angle of a turn, which means that hard braking might be required and could cause extra danger.

The higher the speed that could be maintained in a turn, the lower the number of crashes, Figure \ref{fig:results3} describes. This can be explained by the descreased speed differences for a car, which doesn't cause head-tail collisions in braking anymore.

By using the circuit model with 5 rows, an attempt is been made to copy the possibilities in a circuit. Though, as just said, turns have smaller widths, and also the first straight differs per circuit in general. These extra constrains could give drivers more options and could influence the total crash model.

Next to circuit limitations, accelerations could be improved as well. In the simulation, we used the same number for acceleration per car for the whole season. As cars are being modified all the time, and drivers shift different every single time, it's hard to take this into account. The numbers being used are based on a selected set of cars, as no other information was available. In this, the acceleration of the first 100 metres is used, although it's clear that acceleration declines as the speed goes up.

The last major thing that influences our research is the way of seeing crashes. Wherein the model we now randomized head-tail collisions, if cars came close enough, and there weren't other options, also side-side collisons could be added.

Despite, or maybe due, all these various (missing) constraints, no hard correlations in the result can be found among speeds and (brake) distances. E.g. when a turn is far away from the start, but needs a low speed, with long braking distance, it could have the same outcome as when this turn is right after start. This tells us that we found no hard (safety) reasons why the FIA prefers to have a turn at least 250 metres.

By improving the model, with futher research, and creating a better representation of the real world it might possible to find other correlations. Some last points on weather circumstances and regulations on defending a position show that there is much left to be done. It might be a good idea to improve the study on for example all crashes on one certain circuit, and extend that, instead of trying to compare all circuits at once.

Our research set a step beyond the well-known traffic model and tries to simulate under harder conditions. This attempt wasn't perfect but indicates how the well known Nagel-Schreckenberg model can be used in more difficult situations. Also can be concluded that it's hard to tell which exact combination of properties should be used to have a first turn with few crashes.
