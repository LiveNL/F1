Formula 1 crashes are more a rule than an exception. "In the 1960’s, 1 accident in every 8 in Formula One events resulted in a fatality or serious injury (defined as an injury that prevented the driver from continuing to participate in the event or subsequent events), with some years as high as 1 in 4." \cite{motorsports-safety}. In the Grand Prix' \footnote{Grand Prix (auto racing) an international race for Formula One cars first held in France in 1906 and now staged on a number of circuits around the world [name originally used for the Grand Prix de Paris];\cite{grand-prix}} in 2017 only, half of the races had incidents in the part of the track including the first turn \cite{incidents}.\\
Previous research took various properties of racing into account, such as regulations and safety measurements on technical level of a car to improve the safety of a driver \cite{safety-review}. This research extends this safety impression by looking at circuits. By taking a closer look at the start-section including the first turn in particular, we hope to find an indication that safety can be improved. The Federation Internationale De L'Automobile, FIA, which is responsible for regulations on safety for both circuits and cars state; "There should preferably be at least 250 m between the start line and the first corner." \cite{fia-starting-straight}. We'll have a look at measurements like this distance, braking and speed.\\
In some ways, Formula 1 races are similar to normal traffic flows, which can be simulated using the Nagel-Schreckenberg model \cite{nagel-schreckenberg}. This will be used as main idea for our model with as major difference that cars can crash while taking part in a simulation. Also, where in the original model a single lane is being used, we mimic the design of a Formula 1 circuit.\\

