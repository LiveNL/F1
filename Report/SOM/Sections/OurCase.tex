In our simulation we used Ruby's pseudo-random number generator\cite{ruby-rng} which is a modified Mersenne Twister\cite{mersenne-twister} with a period of 2**19937-1.  To test the distribution of the random number generator, we will use it as die.

The null and alternative hypothesis:
\begin{itemize}
	\itemsep0em
	\item H\textsubscript{0}: The random number generator is a good generator
	\item H\textsubscript{1}: The random number generator is a bad generator
\end{itemize}

Other necessary values:
\begin{itemize}
	\itemsep0em
  \item the level of significance 0.05$\alpha$ (same as example-value);
	\item the \(d.f. = 6 - 1 = 5\);
	\item and the critical value, found in table \ref{table:chisquare}, \textbf{11.07}.
\end{itemize}

The results of the 'die rolls' are shown in table \ref{table:expobssquare2}, with a quantity of 50,000. The sum of the values in the ${\chi}^2$ column is \textbf{1.2834}, which is our ${\chi}^2$.

\bigskip
\begin{tabular}{l|l|l|l}
    \bfseries Nr. & \bfseries Expected & \bfseries Observed & \bfseries ${\chi}^2$% specify table head
    \csvreader[head to column names, separator=semicolon]{expobs2.csv}{}% use head of csv as column names
    {\\\hline\csvcoli&\e&\o&\x}% specify your coloumns here
\end{tabular}
\captionof{table}{Die rolls, 50,000 times with seed: 32614225125840008605808656793674267044.}\label{table:expobssquare2}

\smallskip
Drawing the ${\chi}^2$ value in figure \ref{fig:chisquare} shows, that the value is lower than the critical value. Therefore we can accept H\textsubscript{0}, our random number generator truly generates random numbers.
\vfill\null
