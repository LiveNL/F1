Can we really test a number generator? It's a question that will quite often result in a negative answer. However, it is possible. In this supplementary online material we will descripe the method we used to determine the outcome of our random number generator.

To determine if the random number generator somewhat distributes the random numbers in a fair way, we will need a statistical method. Since that is in our knownlegde the only way. Probably it sounds weird, how it is possible to determine if the random numbers numbers are really random? Even if you use a good die or random number generator, each possible sequence is equally likely to appear. That means that a good random number generator might also produce sequences that look nonrandom and still fail any statistical test, but actually are random. It depends a little bit on the number od samples (or die rolls), because it is prossible that the random number generator produces series of multiple same numbers in a row.

One of the possible methods to statistical determine if the distrubuted random numbers are really random, we will use the hypothesis chi-square (${\chi}^2$) method \cite{chisquareRandom} \cite{chisquare}. This method follows some steps:

\begin{enumerate}
	\itemsep0em
	\item State null (H\textsubscript{0}) and alternative (H\textsubscript{1}) hypothesis
	\item Choose level of significance ($\alpha$)
	\item Find critical values 
	\item Find test statistic
	\item Draw your conclusion
\end{enumerate}


First we will start with an example, in this example we will use the chi-square method to determine if a die from a random casino is a fair die. Since we assume that the die is fair, we can state our \textbf{H\textsubscript{0}} hypothesis: the used die is a fair die. On the other hand, the alternative hypothesis, \textbf{H\textsubscript{1}}, will be that the used die is not a fair die.

The next step is to choose the level of significance. The level of significance, the Greek letter $\alpha$, is the probability of rejecting the null hypothesis, H\textsubscript{0}, when it is true. A significance level of 0,05 means a 5\% risk of concluding that a difference exists when there is not actual difference. The level of 0,05 is quite often chosen for many applications and we will use this number for our example.

After choosing the level of significance, it is time to determine the critical value, sometimes written as c.v., can be determined by using a chi-square table. Before we can lookup the values, we need to calculate the degrees of freedom. The number of degrees of freedom is the number of values in the final calculation of a statistic that are free to vary. Most of the time it is abbreviated as d.f. To calculate the value, we use equation \ref{eq:df}.

\begin{equation}\label{eq:df}
d.f. = n - 1
\end{equation}

For a die, this would result in \(n = 6\) and \(d.f. = 6 - 1 = 5\). Since all the necessary values are now known, we can use table \ref{table:chisquare} to find our critical value. Which will be \textbf{11.07}, according to the given table.\\

\begin{tabular}{l|l|l|l|l|l|l|l|l}
    \bfseries d.f. & \bfseries ${\chi}^2\textsubscript{0.99}$ & \bfseries ${\chi}^2\textsubscript{0.1}$ & \bfseries ${\chi}^2\textsubscript{0.05}$ & \bfseries ${\chi}^2\textsubscript{0.001}$% specify table head
    \csvreader[head to column names, separator=semicolon]{chi-square.csv}{}% use head of csv as column names
    {\\\hline\csvcoli&\a&\e&\f&\h}% specify your coloumns here
\end{tabular}
\captionof{table}{Chi-square critical values.}\label{table:chisquare}

\begin{tikzpicture}
\begin{axis}[grid=both, xtick={0,2,...,20}, ytick={0,0.02,...,0.18}, compat=newest, xlabel=$x$, xlabel style={at={(1,0)}, anchor=west}, ylabel=$pdf f(x)$, ylabel style={rotate=-90,at={(0,1)}, anchor=south}]
\addplot[name path=f, mark=none, red, ultra thick] table [x=x, y=probability, col sep=semicolon] {data.csv};

\path[name path=axis] (axis cs:11.07,0) -- (axis cs:20,0);

\addplot [thick, color=blue, fill=blue, fill opacity=0.5] fill between[of=f and axis, soft clip={domain=11.07:20}];
\addlegendentry{\(\nu=5\)};
\addlegendentry{Reject H\textsubscript{0}};
\end{axis}
\end{tikzpicture}
\captionof{figure}{The ${\chi}^2$ (chi-square) distribution.}\label{fig:chisquare}

In figure \ref{fig:chisquare} we have drawn the line of degrees of freedom and have marked the rejection region, H\textsubscript{0}, based on the critical value.

Now we can calculate the chi-square. The formula for calculating the chi-square is given in \ref{eq:chisquare}, where \textbf{O} is the observed value and \textbf{E} the expected value. By summing all values, we get our test statistic value.

\begin{equation}\label{eq:chisquare}
{\chi}^2=\sum\frac{(O - E)^2}{E}
\end{equation}

The last step in the process is drawing a conclusion. When the test statistic value is larger than the critical value, we can safely reject H\textsubscript{0} and accept H\textsubscript{1}.

\begin{tabular}{l|l|l|l}
    \bfseries Nr. & \bfseries Expected & \bfseries Observed & \bfseries ${\chi}^2$% specify table head
    \csvreader[head to column names, separator=semicolon]{expobs.csv}{}% use head of csv as column names
    {\\\hline\csvcoli&\e&\o&\x}% specify your coloumns here
\end{tabular}
\captionof{table}{Die rolls, 204 times.}\label{table:expobssquare}

Based on the observed and expected data as shown in table \ref{table:expobssquare}, we can calulate the chi-square and will find a sum of \textbf{15.29}. This value is larger than the \textbf{11.07}, so we can reject our H\textsubscript{0} hypothesis and accept H\textsubscript{1}. Therefore the used die is not a fair die.

