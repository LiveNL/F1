Using a statistical method, it's possible to prove that a random number generator distributes in a fair manner. A distribution can be qualified as 'good' when each possible sequence is equally likely to appear. This means that a good random number generator might also produce sequences that look non-random and still fail any statistical test, but actually are random.

One of the possible methods to statistically determine if the random numbers are really random, is the hypothesis Chi-Square (${\chi}^2$) method \cite{chisquareRandom} \cite{chisquare}. This method has the following steps:

\begin{enumerate}
	\itemsep0em
	\item State null (H\textsubscript{0}) and alternative (H\textsubscript{1}) hypothesis.
	\item Choose level of significance ($\alpha$).
	\item Find critical values.
	\item Find test statistic.
	\item Draw your conclusion.
\end{enumerate}

Starting with an example, we'll use the Chi-Square method to determine if a random die can be considered as fair. Since we expect that the die is fair, we can state our \textbf{H\textsubscript{0}} hypothesis: the used die is a fair die. On the other hand, the alternative hypothesis, \textbf{H\textsubscript{1}}, will be: the used die is not a fair die.

The next step is to choose the level of significance. The level of significance, the Greek letter $\alpha$, is the probability of rejecting the null hypothesis, H\textsubscript{0}. A significance level of 0.05 means a 5\% risk of concluding that a difference exists when there is not an actual difference. The level of 0.05 is quite often chosen for many applications and we will use this number for our example.

After choosing the level of significance, a critical value (abbreviated as c.v.) should be determined. This can be done by using a Chi-Square table. This requires the look up of a value corresponding a degrees of freedom. The number of degrees of freedom is the number of values in the final calculation of a statistic that are free to vary. Most of the time it is abbreviated as d.f.. To calculate the value, we use equation \ref{eq:df}.

\begin{equation}\label{eq:df}
d.f. = n - 1
\end{equation}

For a die, this would result in \(n = 6\) and \(d.f. = 6 - 1 = 5\). Since all the necessary values are now known, the table \ref{table:chisquare} can be used to find the critical value. Which will be \textbf{11.07}.\\

\begin{tabular}{l|l|l|l|l|l|l|l|l}
    \bfseries d.f. & \bfseries ${\chi}^2\textsubscript{0.99}$ & \bfseries ${\chi}^2\textsubscript{0.1}$ & \bfseries ${\chi}^2\textsubscript{0.05}$ & \bfseries ${\chi}^2\textsubscript{0.001}$% specify table head
    \csvreader[head to column names, separator=semicolon]{chi-square.csv}{}% use head of csv as column names
    {\\\hline\csvcoli&\a&\e&\f&\h}% specify your coloumns here
\end{tabular}
\captionof{table}{Chi-square critical values. \cite{chisquareTable} }\label{table:chisquare}

\smallskip
\begin{tikzpicture}
\begin{axis}[grid=both, xtick={0,2,...,20}, ytick={0,0.02,...,0.18}, compat=newest, xlabel=$x$, xlabel style={at={(1,0)}, anchor=west}, ylabel=$pdf f(x)$, ylabel style={rotate=-90,at={(0,1)}, anchor=south}]
\addplot[name path=f, mark=none, red, ultra thick] table [x=x, y=probability, col sep=semicolon] {data.csv};

\path[name path=axis] (axis cs:11.07,0) -- (axis cs:20,0);

\addplot [thick, color=blue, fill=blue, fill opacity=0.5] fill between[of=f and axis, soft clip={domain=11.07:20}];
\addlegendentry{\(\nu=5\)};
\addlegendentry{Reject H\textsubscript{0}};
\end{axis}
\end{tikzpicture}
\captionof{figure}{The ${\chi}^2$ (Chi-Square) distribution.}\label{fig:chisquare}
\smallskip

In figure \ref{fig:chisquare} we have drawn the line of degrees of freedom and have marked the rejection region, H\textsubscript{0}, based on the critical value.

Now we can calculate the Chi-Square. The formula for calculating the Chi-Square is given in equation \ref{eq:chisquare}, where \textbf{O} is the observed value and \textbf{E} the expected value. By summing all values, we get the statistic value.

\begin{equation}\label{eq:chisquare}
{\chi}^2=\sum\frac{(O - E)^2}{E}
\end{equation}

The last step in the process is drawing a conclusion. When the test statistic value is larger than the critical value, we can safely reject H\textsubscript{0} and accept H\textsubscript{1}.
\smallskip

\begin{tabular}{l|l|l|l}
    \bfseries Nr. & \bfseries Expected & \bfseries Observed & \bfseries ${\chi}^2$% specify table head
    \csvreader[head to column names, separator=semicolon]{expobs.csv}{}% use head of csv as column names
    {\\\hline\csvcoli&\e&\o&\x}% specify your coloumns here
\end{tabular}
\captionof{table}{Die rolls, 204 times.}\label{table:expobssquare}

\smallskip
Based on the observed and expected data as shown in table \ref{table:expobssquare}, we can calulate the Chi-Square and will find a sum of \textbf{15.29}. This value is larger than the \textbf{11.07}, so we can reject our H\textsubscript{0} hypothesis and accept H\textsubscript{1}. Therefore the used die is not a fair die.
